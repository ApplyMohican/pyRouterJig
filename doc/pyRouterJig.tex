\documentclass[12pt,final]{article}

\usepackage{amsmath}
\usepackage{amssymb}
\usepackage{hyperref}

\newcommand{\codename}[0]{pyRouterJig}
\newcommand{\codever}[0]{{\codename~0.1.0}}
\newcommand{\email}[0]{\texttt{pyrouterjig@lowrielodge.org}}
\newcommand{\mycopyright}[0]{{Copyright 2015 Robert B. Lowrie}}

\begin{document}

\title{\codever}
\author{\mycopyright \\ \email}
\date{}
\maketitle

\section{Overview}

\codename~is a woodworking layout tool for creating box and dovetail
joints.  It has the following features:
\begin{enumerate}
\item Creates templates for the Incra LS Positioner fence.  In fact, this was
  the main motivation for writing \codename.
\item Support for both English and metric unit systems.  Currently, only
  English has been tested, in part because my LS Positioner is the English
  version.  I would be happy to test the metric version if someone would like
  to send me a metric LS Positioner.
\item May be customized for any finger (box joint) or pin / tail (dovetail)
  spacing pattern.  Currently, two patterns are included (``Equally spaced''
  and ``Variable spaced''), which will be described below.
\item \codename~is licensed software, but you're free to use and modify it.
  See Sec. \ref{sec:license} for the license details.  I really appreciate the
  Internet woodworking community.  I have learned a lot from the tips, plans,
  videos, \emph{etc.} that numerous woodworkers, who are much more talented
  woodworkers than I, have contributed.  Maybe this tool can help them.  I
  also hope that by making \codename~freely available, including the source
  code, others that are more artistic than I can devise extensions and
  improvements to \codename.  By making this software freely available, I am
  not implying that others should do the same for their hard work. 
   Woodworking is my hobby, not my career.  But because
  \codename~is free, as the saying goes, ``you get what you pay for.''  I will
  consider implementing changes and feature requests you might have, but in
  the end, if you desperately want to change the code, learn
  \href{http://www.python.org}{Python} and make the changes yourself!  I
  certainly welcome other \href{http://www.python.org}{Python} developers to
  help improve \codename.
\end{enumerate}

\section{Installation}

To develop the code for this project, all I have access to is a Mac, and in
particular, OSX 10.9.2.\footnote{I'm looking for volunteers to test on other
  platforms, such as Windows and Linux.}  \codename~depends upon the following
\href{http://www.python.org}{Python} packages, which must be installed in
order to run \codename:
\begin{enumerate}
\item \href{http://www.python.org}{Python}.  Python is installed by default on
  the Mac, but I use a slighly different version, as I will discuss below.
\item \href{http://www.matplotlib.org}{Mathplotlib}.  This package is used for
  drawing the joints and Incra template.
\item \href{http://www.wxpython.org}{wxPython}.  This package is used as the
  graphical user interface (GUI).
\end{enumerate}
I install all of these packages using
\href{http://www.anaconda.org}{Anaconda}, which is also available for Windows
and Linux.  I highly recommend Anaconda, as the packages above may have other
dependencies that Anaconda also takes care of installing.

\section{Intervals}

\codename~does all of its computations in terms of what \codename~refers to
as ``intervals.''  By default,
\begin{equation*}
  \text{1 interval} =
  \begin{cases}
    1/32" & \text{for Enligh units},\\
    1 \text{~mm} & \text{for metric}.
  \end{cases}
\end{equation*}
An interval is the resolution of \codename.  All dimensions used by \codename,
such as the router-bit width, are rounded to the nearest number of intervals.
The reason for the default choices above is that these are the resolutions of
the respective Incra LS Positioner fence.  By using intervals, we ensure that it's
possible to position the fence at the exact location desired.  More generally, using
intervals (or ``integer arithmetic'') means that \codename~does not need to be
worried about floating-point errors.

\section{Input Parameters}

Throughout the documentation of \codename, we refer to a ``finger'' not
only as the traditional finger of a box joint, but also generically to refer to a
pin or tail of a dovetail joint.

\codename~has the following input parameters, for any finger spacing
algorithm:
\begin{itemize}
\item \textbf{Board Width [inches or mm]:} The width of the board for the joint.
\item \textbf{Bit Width [inches or mm]:}  The maximum cutting width of the router bit.
\item \textbf{Bit Depth [inches or mm]:} The cutting depth of the router bit.
\item \textbf{Bit Angle [degrees]:} The angle of the router bit.  Zero indicates
  a straight bit (box joint).  Fractional values must be input as a
  floating-point number, such as ``7.5''.
\end{itemize}
These parameters are changed by entering text for the dimension in their
respective text boxes.  In order for the figure to update using the new value
(or values), either the ``Enter'' key must be pressed within a text box, or
the ``Draw'' button must be pressed.

The ``inches'' length may be specified as either a fraction or decimal.  For example,
the following are equivalent:
\begin{itemize}
\item 7 1/2
\item 7.5
\item 7 1 / 2
\item 7 1 /2
\item 7 1/ 2
\end{itemize}
There are currently two finger spacing algorithms:
\begin{itemize}
\item \textbf{Equal:} In this case, the fingers are equally spaced.
  There are three inputs that affect this algorithm:
  \begin{enumerate}
  \item \textbf{B-spacing:} This slider allows you to specify additional spacing between
    the Board-B fingers.
  \item \textbf{Width:} This slider allows you to specify additional width added
    to both Board-A and Board-B fingers.
  \item \textbf{Centered:} This input is only available for straight bits (bit
    angle = 0).  If this box is checked, a finger is always centered on
    the board.  Otherwise, a full finger is started on the left edge, which
    will result in a centerd finger only if the finger width divides into the
    board width an odd number of times.
  \end{enumerate}
\item \textbf{Variable:} In this case a large finger is centered on the board,
  and the fingers decrease in size proportional to the distance to the center.
  There is one input that affects this algorithm:
  \begin{enumerate}
  \item \textbf{Fingers:} This slider allows you to specify the number of
    fingers.  At its minimum value, the width of the center finger is maximized. At
    its maximum value, the width of the center finger is minimized, and the result is
    the same as equally-spaced with, zero ``B-spacing'', zero ``Width'', and
    the ``Centered'' option checked.
  \end{enumerate}
\end{itemize}

\section{Improvements Needed}

I am looking for help to make the improvements outlined in this section.  I
will certainly give credit to those who help make improvements.

\subsection{Windows and Linux Support}

If you can help test and improve \codename~on Windows or Linux platforms,
please contact me.  Ideally, you know Python and can send me proposed patches
(or pull requests on Github).

\subsection{Known Bugs}

\begin{enumerate}
\item The first use of Variable Spacing puts its slider (for number of
  fingers) in the upper left-hand
  corner.  Resizing the window fixes the slider placement.
\item Autosizing after changing the board width is broken.  Again, resizing
  the window should fix this issue, and it appears related to the previous bug.
\item I can't add tic labels to the sliders.  This may be an issue just on
  the Mac.  I actually can add the tic labels, it's just that they don't
  ``hide'' properly.
\end{enumerate}

\subsection{Features}

I'm working on the following features, but would appreciate help:

\begin{enumerate}
\item Enable use of arrow keys for sliders.
\item More inline help, such as define wxPython ``tooltips'' for each input
  value.  These are windows that pop up when you put the mouse over a value.
\item Define the option for ``fold-over templates'' that are appropriate for
  hand-cut joints.  This would be an alternative to the Incra template.
\item More friendly error messages and handling.
\item Get rid of the dependence on matplotlib and use wxPython drawing
  directly.  The biggest issue I've had is printing.  For example, with my
  anaconda installation, wxPython's demo/PrintFramework.py segfaults when I
  try to print.
\item More spacing options.
\item An interactive GUI to allow manual adjustment of fingers positions and
  widths.
\item Consider relaxing the requirement that board and bit dimensions be exact multiples
of intervals.
\end{enumerate}


\section{Software License and Disclaimer}
\label{sec:license}

\codename~is free software: you can redistribute it and/or modify it under
the terms of the \textsc{gnu} General Public License as published by the Free Software
Foundation, either version 3 of the License, or (at your option) any later
version.\\

\codename~is distributed in the hope that it will be useful, but
\textsc{without any warranty}; without even the implied warranty of
\textsc{merchantability} or \textsc{fitness for a particular
  purpose}.  See the \textsc{gnu} General Public License for more details.\\

You should have received a copy of the \textsc{gnu} General Public License along with
\codename, in the file named \textsc{copying}.  If not, see
\url{http://www.gnu.org/licenses/}.\\

Because of the well-known dangers of using woodworking tools, please also read
the following: The authors of \codename~are not responsible for any injury,
death, or financial loss that could conceivably be caused by using \codename.  As a user
of \codename, only you are responsible for using the output of
\codename~safely and responsibly.  Woodworking is very dangerous, and in
particular, using a router or other tools to cut joints is dangerous.  You can
loose your eyesight, fingers, and suffer other serious, even fatal,
injuries. The templates and patterns for joints that \codename~ generate are
simply suggestions for joints.  There is no implication that the joints can be
safely cut with your tools (even if you have a Festool router).  You are
solely responsible for operating your tools in a safe manner.  If you feel
\codename~could conceivably encourage you to operate your tools in an unsafe
manner, don't use \codename.  \codename~does no stress or strength analysis of
the joints it generates (although we would happily accept code changes that
would do such analysis), and there is no implication of any particular joint's
strength.  The joint may fail, even if you glue it up properly. \codename~is
not responsible for joint failures.  There is no guarantee that \codename~is
accurate or even generates a feasible joint.  \codename~will likely generate a
joint pattern that does not fit and destroy your otherwise perfect woodworking
project.  In summary, if you believe, or you believe that your survivors might
believe, that it is even remotely possible that \codename~could cause you and
your family serious bodily injury or death, break your tools, or ruin your
woodworking project, do not use \codename.  You have been warned.

\end{document}
